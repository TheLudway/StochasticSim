\documentclass[12pt]{article}

\usepackage{graphicx}
\usepackage{amsmath}
\usepackage[margin=1in]{geometry}
\usepackage{fancyhdr}
\usepackage{enumerate}
\usepackage[shortlabels]{enumitem}
\usepackage[spanish]{babel}
\usepackage{xurl}
\usepackage{tcolorbox}
\usepackage{titlesec}

\titleclass{\subsubsubsection}{straight}[\subsection]

\newcounter{subsubsubsection}[subsubsection]
\renewcommand\thesubsubsubsection{\thesubsubsection.\arabic{subsubsubsection}}
\renewcommand\theparagraph{\thesubsubsubsection.\arabic{paragraph}} % optional; useful if paragraphs are to be numbered

\titleformat{\subsubsubsection}
  {\normalfont\normalsize\bfseries}{\thesubsubsubsection}{1em}{}
\titlespacing*{\subsubsubsection}
{0pt}{3.25ex plus 1ex minus .2ex}{1.5ex plus .2ex}

\makeatletter
\renewcommand\paragraph{\@startsection{paragraph}{5}{\z@}%
  {3.25ex \@plus1ex \@minus.2ex}%
  {-1em}%
  {\normalfont\normalsize\bfseries}}
\renewcommand\subparagraph{\@startsection{subparagraph}{6}{\parindent}%
  {3.25ex \@plus1ex \@minus .2ex}%
  {-1em}%
  {\normalfont\normalsize\bfseries}}
\def\toclevel@subsubsubsection{4}
\def\toclevel@paragraph{5}
%\def\toclevel@paragraph{6}
\def\toclevel@subparagraph{6}
\def\l@subsubsubsection{\@dottedtocline{4}{7em}{4em}}
\def\l@paragraph{\@dottedtocline{5}{10em}{5em}}
\def\l@subparagraph{\@dottedtocline{6}{14em}{6em}}
\makeatother

\setcounter{secnumdepth}{4}
\setcounter{tocdepth}{4}

% Set up headers and footers
\pagestyle{fancy}
\fancyhf{}  % Clear previous settings

\fancyhead[L]{Alvarado Ludwig - Vera Julián}
\fancyhead[C]{Simulación Estocástica}
\fancyhead[R]{9 de Marzo de 2025}

\fancyfoot[C]{\thepage}
\fancyfoot[C]{\footnotesize Este trabajo está bajo una licencia CC 4.0. Más info: \url{https://creativecommons.org/licenses/by/4.0/}}

\renewcommand{\headrulewidth}{0.2pt}

\begin{document}

\tableofcontents


\section{\textit{Events and probability}}
\subsection{Punto a}
\subsubsection{Solución}

\subsection{Punto b}
\subsubsection{Solución}

\section{\textit{Congruential generators}}
\subsection{Solución}

\section{\textit{Uniformity and independence of the unif}}
\subsection{Solución}

\section{\textit{Inverse method for a discrete r.v}}

\subsection{Punto a}

\subsection{Punto b}

\subsection{Punto c}

\subsection{Punto d}

\subsection{Punto e}

\subsection{Punto f}


\section{\textit{Inverse method for continuos r.v}}

\subsection{Punto a}

\subsection{Punto b}

\subsection{Punto c}

\subsection{Punto d}

\subsection{Punto e}

\section{\textit{Monte Carlo Integration}}
\subsection{Solución}

\section{\textit{Estimating} \(\pi\)}

\subsection{Punto a}

\subsection{Punto b}

\subsection{Punto c}

\subsection{Punto d}

\subsection{Punto e}


\section{\textit{Estimating expected values with Monte Carlo} \(\pi\)}

\subsection{Punto a}

\subsection{Punto b}

\subsection{Punto c}

\subsection{Punto d}

\subsection{Punto e}



\end{document}
